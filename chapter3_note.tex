\documentclass[12pt,a4paper]{article}
\usepackage{ctex}
\usepackage{amsmath,amssymb}
\usepackage{geometry}
\usepackage{booktabs}
\usepackage{graphicx}
\geometry{margin=2.5cm}

\title{《Robotics》第三章}
\author{}
\date{}

\begin{document}
\maketitle

\section{引言:为什么需要几何建模}
在前一章中,我们学到了如何用齐次变换矩阵描述物体的“整体”位置和方向。  
但在机器人学中,我们经常需要更细致地表示:
\begin{itemize}
  \item 某一点的位置;
  \item 某个方向(比如机械臂的轴向);
  \item 两个坐标系之间的相对关系;
  \item 末端执行器在空间中的姿态。
\end{itemize}

这些都离不开一种工具——\textbf{向量(Vector)}。

---

\section{向量的基本定义}
\subsection{几何意义}
一个向量是一个既有“方向”又有“长度”的量,通常画成一根带箭头的线段。  
记为:
\[
\boldsymbol{v} = \overrightarrow{AB}
\]
表示从点 \(A\) 到点 \(B\) 的有向线段。

\subsection{代数表示}
在三维笛卡尔坐标系中,任何向量都可以分解为 $x,y,z$ 三个方向的分量:
\[
\boldsymbol{v} = v_x\,\boldsymbol{i} + v_y\,\boldsymbol{j} + v_z\,\boldsymbol{k}
\]
其中:
\(\boldsymbol{i}, \boldsymbol{j}, \boldsymbol{k}\) 是三个互相垂直的单位向量。

用矩阵形式表示为:
\[
\boldsymbol{v} = 
\begin{bmatrix}
v_x \\ v_y \\ v_z
\end{bmatrix}
\]

---

\section{向量的基本运算}
\subsection{加法与减法}
设有两个向量:
\[
\boldsymbol{a} =
\begin{bmatrix}
a_x \\ a_y \\ a_z
\end{bmatrix},
\quad
\boldsymbol{b} =
\begin{bmatrix}
b_x \\ b_y \\ b_z
\end{bmatrix}
\]
则:
\[
\boldsymbol{a} + \boldsymbol{b} =
\begin{bmatrix}
a_x + b_x \\ a_y + b_y \\ a_z + b_z
\end{bmatrix},
\quad
\boldsymbol{a} - \boldsymbol{b} =
\begin{bmatrix}
a_x - b_x \\ a_y - b_y \\ a_z - b_z
\end{bmatrix}
\]

\subsection{数乘}
向量与标量相乘,只改变长度,不改变方向:
\[
k\boldsymbol{a} =
\begin{bmatrix}
k a_x \\ k a_y \\ k a_z
\end{bmatrix}
\]

\subsection{点积(内积)}
定义为:
\[
\boldsymbol{a}\cdot\boldsymbol{b} = a_x b_x + a_y b_y + a_z b_z
\]
几何意义:
\[
\boldsymbol{a}\cdot\boldsymbol{b} = |\boldsymbol{a}||\boldsymbol{b}|\cos\theta
\]
其中 $\theta$ 为两向量夹角。

由此可得:
\[
\boxed{\cos\theta = \dfrac{\boldsymbol{a}\cdot\boldsymbol{b}}{|\boldsymbol{a}||\boldsymbol{b}|}}
\]
这可以用来求两个方向之间的夹角。

\subsection{叉积(外积)}
叉积结果仍是一个向量,方向垂直于两向量所在平面(遵守右手定则):
\[
\boldsymbol{a}\times\boldsymbol{b} =
\begin{bmatrix}
a_y b_z - a_z b_y \\
a_z b_x - a_x b_z \\
a_x b_y - a_y b_x
\end{bmatrix}
\]
几何意义:
\[
|\boldsymbol{a}\times\boldsymbol{b}| = |\boldsymbol{a}|\,|\boldsymbol{b}|\,\sin\theta
\]
表示由 $\boldsymbol{a}$ 与 $\boldsymbol{b}$ 构成的平行四边形的面积。

---

\section{点的位置向量与坐标变换}
\subsection{位置向量}
在一个坐标系 $O-XYZ$ 中,点 $P$ 的位置可由向量表示:
\[
\boldsymbol{r}_P = x\,\boldsymbol{i} + y\,\boldsymbol{j} + z\,\boldsymbol{k}
\]
或简写为列向量:
\[
\boldsymbol{r}_P =
\begin{bmatrix}
x \\ y \\ z
\end{bmatrix}
\]

\subsection{两点间的向量}
若有两点 $A(x_A, y_A, z_A)$ 和 $B(x_B, y_B, z_B)$,  
则:
\[
\overrightarrow{AB} = \boldsymbol{r}_B - \boldsymbol{r}_A =
\begin{bmatrix}
x_B - x_A \\ y_B - y_A \\ z_B - z_A
\end{bmatrix}
\]

---

\section{不同坐标系之间的向量表示}
\subsection{旋转矩阵的几何意义}
假设有两个坐标系:
\begin{itemize}
  \item 原坐标系:$O-XYZ$
  \item 新坐标系:$O'-X'Y'Z'$
\end{itemize}
它们之间仅有旋转,没有平移。

若新坐标系的三个单位向量在旧坐标系中的坐标为:
\[
\boldsymbol{x}' = 
\begin{bmatrix} r_{11} \\ r_{21} \\ r_{31} \end{bmatrix},\quad
\boldsymbol{y}' = 
\begin{bmatrix} r_{12} \\ r_{22} \\ r_{32} \end{bmatrix},\quad
\boldsymbol{z}' = 
\begin{bmatrix} r_{13} \\ r_{23} \\ r_{33} \end{bmatrix}
\]
则旋转矩阵定义为:
\[
R =
\begin{bmatrix}
r_{11} & r_{12} & r_{13}\\
r_{21} & r_{22} & r_{23}\\
r_{31} & r_{32} & r_{33}
\end{bmatrix}
\]
矩阵的每一列就是新坐标系各轴在旧坐标系中的方向余弦。

\subsection{向量的坐标变换}
设同一个向量 $\boldsymbol{v}$ 在两个坐标系中的表示分别为:
\[
\boldsymbol{v}_A =
\begin{bmatrix} v_{xA}\\ v_{yA}\\ v_{zA} \end{bmatrix}, \quad
\boldsymbol{v}_B =
\begin{bmatrix} v_{xB}\\ v_{yB}\\ v_{zB} \end{bmatrix}
\]
若坐标系 $B$ 相对于 $A$ 发生了旋转 $R_{A}^{B}$,  
则有:
\[
\boxed{
\boldsymbol{v}_A = R_{A}^{B} \boldsymbol{v}_B
}
\]
这就是\textbf{向量坐标变换公式}。  
其反变换(从 $A$ 到 $B$)为:
\[
\boldsymbol{v}_B = (R_{A}^{B})^T \boldsymbol{v}_A
\]
因为旋转矩阵是正交矩阵,满足:
\[
R^T R = I
\]

---

\section{点的坐标变换(带平移)}
若两个坐标系不仅有旋转,还有平移:
\[
\text{坐标系 } B \text{ 相对于 } A \text{ 平移 } \boldsymbol{d} =
\begin{bmatrix} d_x \\ d_y \\ d_z \end{bmatrix}
\]
则任意点 $P$ 的位置满足:
\[
\boldsymbol{r}_P^A = R_{A}^{B} \boldsymbol{r}_P^B + \boldsymbol{d}
\]
这就是上一章齐次矩阵关系的核心部分。

---

\section{例题:两坐标系变换}
假设坐标系 $B$ 相对 $A$:
- 绕 $z$ 轴旋转 $90^\circ$;
- 再平移 $(2,0,0)$。

\noindent
旋转矩阵:
\[
R_{A}^{B} =
\begin{bmatrix}
0 & -1 & 0\\
1 & 0 & 0\\
0 & 0 & 1
\end{bmatrix}, \quad
\boldsymbol{d} =
\begin{bmatrix}
2 \\ 0 \\ 0
\end{bmatrix}
\]
若点 $P$ 在 $B$ 系中坐标为 $(1,1,0)$,则其在 $A$ 系中的坐标为:
\[
\boldsymbol{r}_P^A =
R_{A}^{B}\boldsymbol{r}_P^B + \boldsymbol{d}
=
\begin{bmatrix}
0 & -1 & 0\\
1 & 0 & 0\\
0 & 0 & 1
\end{bmatrix}
\begin{bmatrix}
1 \\ 1 \\ 0
\end{bmatrix}
+
\begin{bmatrix}
2 \\ 0 \\ 0
\end{bmatrix}
=
\begin{bmatrix}
1 \\ 1 \\ 0
\end{bmatrix}
\]
结果表示该点在 $A$ 系中坐标为 $(1,1,0)$。

---

\section{总结}
\begin{enumerate}
  \item 向量用于描述空间中的位置与方向;
  \item 向量的加减、点积、叉积分别表示合成、夹角与垂直面积;
  \item 坐标变换通过旋转矩阵 $R$ 与平移向量 $\boldsymbol{d}$ 实现;
  \item 两坐标系之间的关系可统一写作:
  \[
  \boldsymbol{r}_P^A = R_{A}^{B}\boldsymbol{r}_P^B + \boldsymbol{d}
  \]
  \item 旋转矩阵是正交矩阵,满足 $R^T R = I$;
  \item 该章为后续“正运动学与逆运动学”打下数学基础。
\end{enumerate}

\end{document}
