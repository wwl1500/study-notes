\documentclass[12pt,a4paper]{article}
\usepackage{ctex}
\usepackage{amsmath,amssymb}
\usepackage{geometry}
\usepackage{booktabs}
\usepackage{graphicx}
\geometry{margin=2.5cm}

\title{《Robotics》第四章}
\author{}
\date{}

\begin{document}
\maketitle

\section{引言:什么是正运动学}
机械臂有多个关节和连杆,每个关节可以转动或滑动。  
所谓“\textbf{正运动学}(Forward Kinematics)”就是——  
\begin{quote}
已知每个关节的运动参数(角度或位移),求出末端执行器的位置与方向。
\end{quote}

\noindent
相对地,“逆运动学”是已知末端位置,求各关节参数。

\paragraph{举个例子:}
假设一个两连杆机械臂(类似人类的肩膀和手肘):
- 第一个关节角为 $\theta_1$;
- 第二个关节角为 $\theta_2$;
- 连杆长度为 $L_1, L_2$;
则末端的位置是:
\[
\begin{cases}
x = L_1\cos\theta_1 + L_2\cos(\theta_1+\theta_2)\\
y = L_1\sin\theta_1 + L_2\sin(\theta_1+\theta_2)
\end{cases}
\]
这就是一个最简单的正运动学问题。

---

\section{Denavit–Hartenberg(DH)建模法}
为了描述任意复杂的机械臂,我们需要一种统一的数学表示。  
1955 年,Denavit 和 Hartenberg 提出了一种方法(简称 \textbf{DH 参数法})。

DH 方法通过为每个连杆定义一个坐标系,用四个参数描述它与前一连杆的几何关系。

---

\section{四个 DH 参数的定义}
考虑第 $i-1$ 与第 $i$ 个连杆之间的连接关系:

| 符号 | 名称 | 含义 |
|------|------|------|
| \(a_i\) | 连杆长度 | $x_i$ 轴在 $x_{i-1}$ 轴方向上的投影长度 |
| \(\alpha_i\) | 连杆扭角 | $z_{i-1}$ 与 $z_i$ 之间的夹角 |
| \(d_i\) | 连杆偏距 | $z_{i-1}$ 轴到 $x_i$ 轴的距离(沿 $z_{i-1}$) |
| \(\theta_i\) | 关节角 | $x_{i-1}$ 与 $x_i$ 之间的夹角(绕 $z_{i-1}$) |

\paragraph{说明:}
- 对于转动关节(R 型),\(\theta_i\) 是变量,\(d_i\) 为常量;
- 对于移动关节(T 型),\(d_i\) 是变量,\(\theta_i\) 为常量。

---

\section{DH 坐标系建立规则}
每个关节分配一个坐标系 $O_i-x_i y_i z_i$:

1. $z_i$ 轴沿关节轴线方向;
2. $x_i$ 轴沿着 $z_{i-1}$ 与 $z_i$ 之间的公法线;
3. 原点 $O_i$ 设在两轴公法线与 $z_i$ 的交点处;
4. $y_i$ 轴由右手定则确定。

这样,每个坐标系之间的几何关系就唯一确定。

---

\section{单个连杆的变换矩阵}
根据 DH 定义,坐标系 $i$ 相对于坐标系 $i-1$ 的齐次变换矩阵为:
\[
^{i-1}T_i =
\begin{bmatrix}
\cos\theta_i & -\sin\theta_i\cos\alpha_i & \sin\theta_i\sin\alpha_i & a_i\cos\theta_i\\
\sin\theta_i & \cos\theta_i\cos\alpha_i & -\cos\theta_i\sin\alpha_i & a_i\sin\theta_i\\
0 & \sin\alpha_i & \cos\alpha_i & d_i\\
0 & 0 & 0 & 1
\end{bmatrix}
\]

\paragraph{推导思路:}
它由以下四个基本变换依次组成:
1. 绕 $z_{i-1}$ 旋转 $\theta_i$;
2. 沿 $z_{i-1}$ 平移 $d_i$;
3. 沿 $x_i$ 平移 $a_i$;
4. 绕 $x_i$ 旋转 $\alpha_i$。

\noindent
将这四步的齐次矩阵相乘即可得到上式。

---

\section{整条机械臂的正运动学}
对于 $n$ 个关节的机械臂,末端坐标系 $n$ 相对基座坐标系 $0$ 的齐次变换矩阵为:
\[
^{0}T_n = {^{0}T_1} {^{1}T_2} {^{2}T_3} \cdots {^{n-1}T_n}
\]

\paragraph{含义:}
这个矩阵的上 $3\times3$ 部分给出末端方向(旋转矩阵 $R$),  
右上角的 $3\times1$ 列给出末端位置(平移向量 $\boldsymbol{p}$):
\[
^{0}T_n =
\begin{bmatrix}
R & \boldsymbol{p}\\
0 & 1
\end{bmatrix}
\]
因此,末端执行器的“姿态(Pose)”完全由 $^{0}T_n$ 决定。

---

\section{二维平面两连杆机械臂示例}
\subsection{几何关系}
设:
\begin{itemize}
  \item 第一连杆长度 $L_1$,关节角 $\theta_1$;
  \item 第二连杆长度 $L_2$,关节角 $\theta_2$;
  \item 所有运动在 $xy$ 平面内。
\end{itemize}

\subsection{位置推导}
从几何上:
\[
\begin{cases}
x = L_1\cos\theta_1 + L_2\cos(\theta_1+\theta_2)\\[6pt]
y = L_1\sin\theta_1 + L_2\sin(\theta_1+\theta_2)
\end{cases}
\]

写成齐次矩阵形式:
\[
^{0}T_1 =
\begin{bmatrix}
\cos\theta_1 & -\sin\theta_1 & 0 & L_1\cos\theta_1\\
\sin\theta_1 & \cos\theta_1 & 0 & L_1\sin\theta_1\\
0 & 0 & 1 & 0\\
0 & 0 & 0 & 1
\end{bmatrix},
\quad
^{1}T_2 =
\begin{bmatrix}
\cos\theta_2 & -\sin\theta_2 & 0 & L_2\cos\theta_2\\
\sin\theta_2 & \cos\theta_2 & 0 & L_2\sin\theta_2\\
0 & 0 & 1 & 0\\
0 & 0 & 0 & 1
\end{bmatrix}
\]

两者相乘:
\[
^{0}T_2 = {^{0}T_1}{^{1}T_2}
\]
化简后:
\[
^{0}T_2 =
\begin{bmatrix}
\cos(\theta_1+\theta_2) & -\sin(\theta_1+\theta_2) & 0 & L_1\cos\theta_1 + L_2\cos(\theta_1+\theta_2)\\
\sin(\theta_1+\theta_2) & \cos(\theta_1+\theta_2) & 0 & L_1\sin\theta_1 + L_2\sin(\theta_1+\theta_2)\\
0 & 0 & 1 & 0\\
0 & 0 & 0 & 1
\end{bmatrix}
\]
这给出了末端的完整位姿。

---

\section{三维例子:RRR 机械臂(空间三转动)}
三自由度仿人型机械臂(RRR 型)通常每个关节都是转动的。  
它的末端姿态由三个角度(常称“Euler angles”或“roll–pitch–yaw”)决定。

对应的总变换:
\[
^{0}T_3 = {^{0}T_1}{^{1}T_2}{^{2}T_3}
\]
通过给出各关节的 DH 参数,可依次求出每个 $^{i-1}T_i$ 并相乘。

---

\section{物理意义与结论}
\begin{itemize}
  \item 正运动学求解了“关节变量 → 末端姿态”的映射;
  \item 齐次变换矩阵将所有旋转与平移统一在一个框架;
  \item DH 参数提供了系统化的建模方法;
  \item 任意多连杆机械臂的位姿可通过矩阵连乘得到;
  \item 该结果是运动学、雅可比矩阵和控制分析的基础。
\end{itemize}

---

\section{总结公式表}

| 概念 | 表达式 |
|------|----------|
| 单连杆变换 | \(^{i-1}T_i = R_z(\theta_i)T_z(d_i)T_x(a_i)R_x(\alpha_i)\) |
| 多连杆合成 | \(^{0}T_n = \prod_{i=1}^n\,^{i-1}T_i\) |
| 末端位置 | 右上角 \(3\times1\) 向量 |
| 末端姿态 | 左上角 \(3\times3\) 旋转矩阵 |
| 二连杆平面臂 | \(x=L_1\cos\theta_1+L_2\cos(\theta_1+\theta_2)\), \\ \(y=L_1\sin\theta_1+L_2\sin(\theta_1+\theta_2)\) |

---

\end{document}
