\documentclass[12pt,a4paper]{article}
\usepackage{ctex}
\usepackage{amsmath,amssymb}
\usepackage{geometry}
\usepackage{graphicx}
\usepackage{booktabs}
\usepackage{hyperref}
\geometry{margin=2.5cm}

\title{《Robotics》第二章}
\date{}

\begin{document}
\maketitle

\section{为什么要引入齐次变换矩阵}
在机器人学中,机械臂的每一段(连杆)都可以视为一个刚体。  
描述刚体在空间中的位置与姿态,需要同时考虑:
\begin{itemize}
  \item \textbf{平移}(translation):描述物体在空间中的位置;
  \item \textbf{旋转}(rotation):描述物体的方向。
\end{itemize}

但在实际计算中,我们希望能用\textbf{一个统一的数学表达式}来同时处理平移和旋转。  
这就是为什么引入——**齐次变换矩阵(Homogenous Transformation Matrix)**。

齐次矩阵将旋转和平移统一为一个 \(4\times4\) 矩阵运算,使得:
\[
\text{旋转 + 平移 = 一次矩阵相乘}
\]

---

\section{平移变换的数学表示}
设有一个点 \(P\),其在坐标系中的位置为:
\[
P = 
\begin{bmatrix}
x \\ y \\ z
\end{bmatrix}
\]

如果我们让这个点平移一个向量
\[
\boldsymbol{d} = 
\begin{bmatrix}
a \\ b \\ c
\end{bmatrix}
\]
则平移后的点 \(P'\) 坐标为:
\[
P' = 
\begin{bmatrix}
x + a \\ y + b \\ z + c
\end{bmatrix}
\]

这种表达很直观,但无法与旋转统一起来。  
于是我们引入一个“**齐次坐标**”,也就是在坐标向量末尾加上一个 1:
\[
\boldsymbol{q} = 
\begin{bmatrix}
x \\ y \\ z \\ 1
\end{bmatrix}
\]

这样,平移就可以写成一个矩阵乘法:
\[
\boldsymbol{v} =
\underbrace{
\begin{bmatrix}
1 & 0 & 0 & a \\
0 & 1 & 0 & b \\
0 & 0 & 1 & c \\
0 & 0 & 0 & 1
\end{bmatrix}
}_{\text{平移矩阵 } T(a,b,c)}
\begin{bmatrix}
x \\ y \\ z \\ 1
\end{bmatrix}
=
\begin{bmatrix}
x+a \\ y+b \\ z+c \\ 1
\end{bmatrix}
\]

我们称这个 \(4\times4\) 矩阵为\textbf{平移齐次矩阵}:
\[
T(a,b,c) = 
\begin{bmatrix}
1 & 0 & 0 & a \\
0 & 1 & 0 & b \\
0 & 0 & 1 & c \\
0 & 0 & 0 & 1
\end{bmatrix}
\]

---

\section{旋转变换的推导}
旋转描述物体绕坐标轴的转动。  
我们采用右手定则:  
右手大拇指指向旋转轴正方向,四指弯曲方向为正旋转方向。

\subsection{绕 $x$ 轴旋转}
如图所示,旋转角度为 $\alpha$。  
原点不动,$x$ 轴方向不变,$y$ 轴和 $z$ 轴发生变化:
\[
\begin{cases}
y' = y\cos\alpha - z\sin\alpha \\
z' = y\sin\alpha + z\cos\alpha
\end{cases}
\]
写成矩阵形式:
\[
R_x(\alpha) =
\begin{bmatrix}
1 & 0 & 0 & 0\\
0 & \cos\alpha & -\sin\alpha & 0\\
0 & \sin\alpha & \cos\alpha & 0\\
0 & 0 & 0 & 1
\end{bmatrix}
\]

\subsection{绕 $y$ 轴旋转}
绕 $y$ 轴旋转角度 $\beta$:
\[
\begin{cases}
x' = x\cos\beta + z\sin\beta \\
z' = -x\sin\beta + z\cos\beta
\end{cases}
\]
齐次矩阵表示:
\[
R_y(\beta) =
\begin{bmatrix}
\cos\beta & 0 & \sin\beta & 0\\
0 & 1 & 0 & 0\\
-\sin\beta & 0 & \cos\beta & 0\\
0 & 0 & 0 & 1
\end{bmatrix}
\]

\subsection{绕 $z$ 轴旋转}
绕 $z$ 轴旋转角度 $\gamma$:
\[
\begin{cases}
x' = x\cos\gamma - y\sin\gamma \\
y' = x\sin\gamma + y\cos\gamma
\end{cases}
\]
齐次矩阵为:
\[
R_z(\gamma) =
\begin{bmatrix}
\cos\gamma & -\sin\gamma & 0 & 0\\
\sin\gamma & \cos\gamma & 0 & 0\\
0 & 0 & 1 & 0\\
0 & 0 & 0 & 1
\end{bmatrix}
\]

---

\section{综合旋转和平移:齐次变换矩阵}
现在我们希望一个矩阵能同时描述物体的“旋转 + 平移”。  
定义:
\[
H =
\begin{bmatrix}
R & \boldsymbol{d} \\
0\ 0\ 0 & 1
\end{bmatrix}
\]
其中:
\begin{itemize}
  \item \(R\) 是 \(3\times3\) 的旋转矩阵;
  \item \(\boldsymbol{d} = [a\ b\ c]^T\) 是平移向量;
\end{itemize}

那么,对于任意点
\[
\boldsymbol{p} = 
\begin{bmatrix}
x \\ y \\ z \\ 1
\end{bmatrix}
\]
其经过旋转和平移后的新坐标为:
\[
\boldsymbol{p}' = H \boldsymbol{p}
\]

即:
\[
\boxed{
\boldsymbol{p}' = R\boldsymbol{p} + \boldsymbol{d}
}
\]
这就是机器人学中描述刚体位置和方向的\textbf{基本方程}。

---

\section{齐次矩阵的物理含义}
\subsection{表示姿态(Pose)}
矩阵 $H$ 不仅仅是一个变换运算符,它也代表一个坐标系相对于另一个坐标系的姿态:
\[
H =
\begin{bmatrix}
R & \boldsymbol{d} \\
0 & 1
\end{bmatrix}
\]
含义:
- \(R\):新坐标系的三个轴方向在旧坐标系中的投影;
- \(\boldsymbol{d}\):新坐标系原点在旧坐标系中的位置。

\subsection{多个变换的组合}
如果一个坐标系先经历 $H_1$ 变换,再经历 $H_2$ 变换,  
则总变换为:
\[
H = H_1 H_2
\]
这叫做“**齐次变换的链式法则**”,  
在机器人机械臂建模中极其重要。

---

\section{例子:连续变换}
假设一个坐标系先绕 $z$ 轴旋转 90°,再平移 $(4,-3,7)$。  
写成矩阵:
\[
H = T(4,-3,7)\,R_z(90^\circ)
\]
数值代入:
\[
H =
\begin{bmatrix}
0 & -1 & 0 & 4\\
1 & 0 & 0 & -3\\
0 & 0 & 1 & 7\\
0 & 0 & 0 & 1
\end{bmatrix}
\]

这表示:  
- 新坐标系的 $x'$ 轴指向旧坐标系的 $y$ 方向;  
- 新原点位于 $(4,-3,7)$。

---

\section{总结}
\begin{itemize}
  \item 平移和旋转都可表示为 \(4\times4\) 矩阵;
  \item 齐次变换矩阵 $H$ 将两者统一;
  \item 通过矩阵乘法可以方便地进行多级坐标变换;
  \item 在机器人学中,每个连杆之间的关系都用齐次矩阵描述。
\end{itemize}

\end{document}
