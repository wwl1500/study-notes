\documentclass[12pt,a4paper]{article}
\usepackage[UTF8, scheme=plain]{ctex}        % 中文支持
\usepackage{amsmath,amssymb}
\usepackage{geometry}
\usepackage{graphicx}
\usepackage{booktabs}
\geometry{margin=2.5cm}

% 设置中文字体
\setCJKmainfont{AR PL UMing CN}
\setCJKsansfont{Noto Sans CJK SC}
\setCJKmonofont{AR PL KaitiM GB}

\title{《Robotics》第一章}
\author{}
\date{}

\begin{document}
\maketitle

\section{引言:什么是工业机器人}
根据国际标准 ISO~8373:
\begin{quote}
工业机器人操纵器是一种带有反馈控制的、可重编程的、多用途机械系统,至少具有三个或更多自由度,用于工业自动化。
\end{quote}

\noindent
这句话中包含几个关键概念:
\begin{itemize}
  \item \textbf{反馈控制(feedback control)}:机器人能用传感器感知自己的动作(如位置、速度),并自动修正误差。
  \item \textbf{可重编程(reprogrammable)}:改变程序就能执行不同的任务。
  \item \textbf{多用途(multipurpose)}:同一台机器人可执行焊接、搬运、装配等多种工序。
\end{itemize}

机器人通常由三部分组成:
\[
\text{执行机构(motors)} + \text{传感器(sensors)} + \text{控制系统(controller)}
\]
控制器的目标是让机器人的“手”(末端执行器)准确按照我们设定的轨迹运动。

\section{自由度(Degree of Freedom, DOF)的基本概念}
自由度(简称 DOF)表示一个系统的\textbf{独立运动方向的数量}。  
简单地说:描述这个系统位置所需要的独立数值(坐标)有多少,它就有多少个自由度。

\subsection{从质点讲起}
我们从最简单的情况开始——一个小球(质点):

\paragraph{(1)一维运动:}
如果小球只能在一条直线上来回滑动(比如套在铁丝上),
它的位置可以用一个数 \(x\) 表示。  
所以它有:
\[
\boxed{1\ \text{个自由度}}
\]

\paragraph{(2)二维运动:}
如果小球能在一个平面上移动(比如桌面上滑动),
它的位置由两个坐标 \(x, y\) 表示:
\[
\boldsymbol{r} = x\,\boldsymbol{i} + y\,\boldsymbol{j}
\]
这里 \(\boldsymbol{i},\boldsymbol{j}\) 是坐标轴方向。  
此时我们需要两个数值来确定小球的位置,所以:
\[
\boxed{2\ \text{个自由度}}
\]

\paragraph{(3)三维运动:}
如果小球能在空间中任意运动,则位置向量为:
\[
\boldsymbol{r} = x\,\boldsymbol{i} + y\,\boldsymbol{j} + z\,\boldsymbol{k}
\]
需要三个坐标 \((x,y,z)\) 才能确定位置,因此:
\[
\boxed{3\ \text{个自由度}}
\]

\subsection{刚体的自由度推导}
刚体可以看作由多个质点组成,但这些质点之间的\textbf{距离固定}。  
假设刚体由 3 个不在一条直线上的质点组成。

\begin{itemize}
  \item 每个质点在空间中有 3 个坐标:总共 \(3\times3 = 9\) 个变量。
  \item 由于刚体形状不变,三对质点间距离固定,产生 3 个约束条件:
  \[
  |\boldsymbol{r}_1 - \boldsymbol{r}_2| = c_{12}, \quad
  |\boldsymbol{r}_1 - \boldsymbol{r}_3| = c_{13}, \quad
  |\boldsymbol{r}_2 - \boldsymbol{r}_3| = c_{23}
  \]
  \item 所以独立变量数为 \(9 - 3 = 6\)。
\end{itemize}

\noindent
因此,一个刚体在三维空间中有:
\[
\boxed{6\ \text{个自由度} = 3\ \text{个平移} + 3\ \text{个转动}}
\]

\paragraph{解释:}
\begin{itemize}
  \item \textbf{平移自由度}:沿 $x, y, z$ 三个方向的位置变化;
  \item \textbf{转动自由度}:绕 $x, y, z$ 三个轴的旋转角度;
\end{itemize}
这六个参数完全决定刚体的“姿态(pose)”:
\[
\text{Pose} = \text{位置(position)} + \text{方向(orientation)}
\]

\section{关节与机械臂结构}
工业机器人不是一个整体的刚体,而是由若干\textbf{刚性连杆(links)}通过\textbf{关节(joints)}连接而成。

\subsection{关节类型}
每个关节通常只保留一个自由度:
\begin{itemize}
  \item \textbf{转动关节(Revolute Joint, R)}:允许绕固定轴旋转。  
    变量:转角 \(\theta\)
  \item \textbf{移动关节(Prismatic Joint, T)}:允许沿固定轴滑动。  
    变量:距离 \(d\)
\end{itemize}

\subsection{串联机构自由度计算}
假设机器人有 $J$ 个一自由度关节。  
整个机构包含 $N=J+1$ 个刚体(包括基座)。  
每个刚体原本有 6 个自由度,总共 \(6N\)。  
每个关节去掉 \(6-1=5\) 个自由度(因为只保留 1 个),  
于是系统总自由度为:
\[
F = 6N - 5J = 6(J+1) - 5J = J + 6
\]
若基座固定(减掉 6 个基座自由度),得:
\[
\boxed{F = J}
\]
也就是说,一个有 6 个关节的机械臂,就有 6 个自由度。

\section{机械臂与腕部}
机器人操纵器一般分为三部分:

\begin{itemize}
  \item \textbf{机械臂(Arm)}:主要负责移动末端的位置;
  \item \textbf{腕部(Wrist)}:调整末端的方向;
  \item \textbf{夹具(Gripper)}:夹取或释放物体。
\end{itemize}

为了能在空间中任意“摆放”物体,机械臂 + 腕部至少要提供 6 个独立运动方向:
\[
3\ \text{个平移} + 3\ \text{个转动}
\]

\section{典型机械臂结构}
工业上常见的三自由度机械臂主要有以下几种形式:

\begin{table}[h!]
\centering
\begin{tabular}{llll}
\toprule
名称 & 关节类型 & 自由度组合 & 工作空间形状 \\
\midrule
仿人型 (Anthropomorphic) & R–R–R & 三转动 & 球形 \\
球坐标型 (Spherical) & R–R–T & 两转动一平移 & 球形 \\
SCARA 型 & R–R–T & 两转动一平移 & 圆柱形 \\
圆柱型 (Cylindrical) & R–T–T & 一转动两平移 & 圆柱形 \\
笛卡尔型 (Cartesian) & T–T–T & 三平移 & 长方体 \\
\bottomrule
\end{tabular}
\end{table}

这些结构的选择取决于所需的工作空间形状和任务类型。  
例如:
\begin{itemize}
  \item SCARA 机器人常用于装配;
  \item 笛卡尔型适合精密搬运;
  \item 仿人型可完成复杂空间路径。
\end{itemize}

\section{总结}
\begin{enumerate}
  \item 自由度表示系统能独立运动的方向数量;
  \item 质点在空间有 3 个自由度,刚体有 6 个自由度;
  \item 串联机器人每个关节贡献 1 个自由度;
  \item 完整空间操作需至少 6 个自由度;
  \item 工业机器人由机械臂、腕部和夹具组成;
  \item 各类机械臂结构可归纳为 RRR、RRT、RTT、TTT 等形式。
\end{enumerate}

\end{document}
